% !TEX TS-program = xelatex
% !TEX encoding = UTF-8 Unicode
% !Mode:: "TeX:UTF-8"

\documentclass{resume}
\usepackage{zh_CN-Adobefonts_external} % Simplified Chinese Support using external fonts (./fonts/zh_CN-Adobe/)
%\usepackage{zh_CN-Adobefonts_internal} % Simplified Chinese Support using system fonts
\usepackage{linespacing_fix} % disable extra space before next section
\usepackage{cite}

\begin{document}
\pagenumbering{gobble} % suppress displaying page number

\name{何逸轩}
\centerline{求职意向: nlp 工程师}
% \basicInfo{hyx\_ict@163.com}{(+86) 185-1023-8192}{北京市海淀区中国科学院南路6号}


\basicInfo{
  \email{hyx\_ict@163.com} \textperiodcentered\
  \phone{(+86) 185-1023-8192}
  }

\section{\faGraduationCap\  教育背景}
\datedsubsection{\textbf{中科院计算所}, 计算机理论,硕士}{2015.9 -- 2018.6}
\datedsubsection{\textbf{北京邮电大学}, 软件工程,学士}{2011.9 -- 2015.6}

\section{\faGraduationCap  工作背景}
\datedsubsection{\textbf{微信搜一搜\ 语义计算组}}{2018.7 -- 至今}

\section{\faUsers 项目经历}

\datedsubsection{\textbf{同义词模块}}{2018.7 -- 至今}
\role{微信搜一搜}{项目人数:1}
\begin{onehalfspacing}
对于 query 进行同义词替换,辅助召回和相关性模块
\begin{itemize}
  \item 离线通过用户点击、session 切换等方式挖掘相似 query;在相似 query 基础上构造深度对齐模型,将对齐转化为相似的上下文查找,相比 IBM model2 在长尾对齐结果的准确率上有 5 倍的提升;同时产出对齐片段供文章改写模型使用
  \item 结合搜狗的曝光日志,字词级别的特征,利用编辑的标注数据训练同义词离线判别的 xgb 模型,准确率 85\%,召回86\%
  \item 利用编辑标注的同义词上下文数据,结合 PMI、词向量、QV、语言模型 等特征训练同义词后验 xgb 模型,准确率 89.2 \%
  \item 相关性模型上引入同义词特征,在分档准确率上有 1.5\% 的绝对提升
\end{itemize}
\end{onehalfspacing}

\datedsubsection{\textbf{query 改写}}{2018.7 -- 至今}
\role{微信搜一搜}{项目人数:2}
\begin{onehalfspacing}
针对于公众号、小程序、文章等进行 query 改写,降低 ncr(无点率)
\begin{itemize}
  \item 利用 session 切换, 相似 query 等渠道挖掘改写数据,结合召回结果相关性,权威度等信息利用编辑标注数据训练离线判别模型,准确率 85\%;同时通过用户的在线点击数据利用 T 检验自动进行改写数据的清洗,在线改写对的准确率达到 90\%。
  \item 结合文本、拼音、字形等数据训练公众号改写深度生成式模型,在公众号文本树上搜索,通过用户点击数据训练,结合历史 attention 等多种信息,准确率达到 75\%。
  \item 结合字、词、拼音、字形等特征,利用生成式误差、相关性误差、非必留误差等构造多任务学习模型学习垂搜文本的向量化表示,在线利用向量召回的方式进行长尾串的改写;离线准确率 85\%,在线改写准确率 90\%。
\end{itemize}
\end{onehalfspacing}

\datedsubsection{\textbf{基于强化学习的文本匹配}}{2017.10 -- 2018.4}
\role{毕业设计}{项目人数:1}

\begin{onehalfspacing}
利用强化学习解决文本匹配问题
\begin{itemize}
  \item 将文本匹配建模为匹配矩阵中行走问题,利用值迭代和策略梯度求解
  \item 利用蒙特卡罗树搜索解决局部最优解问题,在公开数据集上优于其他文本匹配方法
  \item 利用 C++ 并行蒙特卡罗树搜索,4线程下加速比达到 0.8
\end{itemize}
\end{onehalfspacing}

\section{\faTrophy\ 学术竞赛}
\datedline{\textit{kaggle Allstate Claims Severity}\quad \quad {43th/3055}}{2016.09}

\section{\faHeartO\ 获奖荣誉}
\datedline{\textit{中国科学院大学}{\quad 北纬通信硕士生奖 \quad 三好学生}}{2015-2018}

\section{\faCogs\ 个人能力}
% increase linespacing [parsep=0.5ex]
\begin{itemize}[parsep=0.5ex]
  \item 熟悉 C++,Python,Shell 等编程语言,熟悉 linux 基本环境
  \item 熟悉常用的深度学习算法以及强化学习算法
  \item 深入阅读 TensorFlow 源代码,熟悉分布式计算
\end{itemize}

\end{document}