% !TEX TS-program = xelatex
% !TEX encoding = UTF-8 Unicode
% !Mode:: "TeX:UTF-8"

\documentclass{resume}
\usepackage{zh_CN-Adobefonts_external} % Simplified Chinese Support using external fonts (./fonts/zh_CN-Adobe/)
%\usepackage{zh_CN-Adobefonts_internal} % Simplified Chinese Support using system fonts
\usepackage{linespacing_fix} % disable extra space before next section
\usepackage{cite}

\begin{document}

\newcommand{\SubItem}[1]{
    {\setlength\itemindent{15pt} \item[-] #1}
}

\pagenumbering{gobble} % suppress displaying page number

\name{何逸轩}

\basicInfo{
  \email{hyx\_ict@163.com} \textperiodcentered\
  \phone{(+86) 185-1023-8192}
  }

\section{\faGraduationCap  教育背景}
\datedsubsection{\textbf{中科院计算所}, 计算机理论,硕士}{2015.9 -- 2018.6}
\datedsubsection{\textbf{北京邮电大学}, 软件工程,学士}{2011.9 -- 2015.6}

\section{\faUsers  工作背景}
\datedsubsection{\textbf{小红书搜索}}{2023.5-至今}
\datedsubsection{\textbf{百度搜索策略部}}{2020.5-2023.5}
\datedsubsection{\textbf{微信搜一搜}}{2018.7-2020.5}

\section{\faCogs 项目经历}
\datedsubsection{\textbf{小红书搜索相关性}}{2023.5 -- 至今}
\role{小红书搜索}{}
\begin{onehalfspacing}
在小红书主要负责优化粗精排搜索相关性
\begin{itemize}
\item \textbf{精排相关性优化}
\SubItem{通过基础模型升级大模型、预训练语料优化等手段,优化预训练模型效果}
\SubItem{结合 chatgpt 等大模型,进行样本的自动化标注、难 query 生成}
\SubItem{通过 vit、对比学习等手段,引入多模相关性,优化封面图匹配算法}
\item \textbf{粗排相关性优化} 
\SubItem{通过引入大规模对比学习、跨GPU跨step组batch等方法,优化粗排双塔预训练模型效果}
\SubItem{通过引入pairwise蒸馏、有监督蒸馏等手段,提升模型的蒸馏效果}
\end{itemize}
\end{onehalfspacing}

\datedsubsection{\textbf{百度搜索相关性}}{2020.5 -- 2023.5}
\role{百度搜索策略部}{}
\begin{onehalfspacing}
先后担任相关性团队-内容建模方向和阿拉丁整页团队-语义相关性方向负责人,负责优化搜索相关性。
\begin{itemize}
  \item \textbf{网页信息建模}。对网页和阿拉丁结果的信息建模。
  \SubItem{自然结果:在自然结果上,结合 qimp、紧密度 等信息,对自然结果的摘要进行抽取,对问答query优化了 nn 的抽取方案;同时对网页中的连续命中情况设计特征优化}
  \SubItem{阿拉丁结果:对于阿拉丁结果,在 summary 抽取上,针对于结构化的 json 信息抽取进行了优化;同时结合阿拉丁本身特点,利用点击 query 提炼卡片需求词}
  \item \textbf{预训练优化}。ernie的预训练是基于大规模的网络数据,和搜索场景有较大的差异,因此我们在搜索场景上进行了进一步探索,主要包括搜索预训练、数据清洗、知识继承、任务导向预训练等。通过预训练优化,在 ernie 上累计有15\%的pnr提升
  \item \textbf{特殊问题处理}。由于 ernie 的训练中无法有些类型数据样本量过小,导致 ernie 本身难以学习相应匹配范式,因此需要对特殊子问题专门处理。
  \SubItem{核心 term 丢失:对于核心 term,主要通过样本构造和挖掘进行。在整页场景下,样本挖掘结合了卡片本身的信息进行优化,解决了 70\% 的历史case}
  \SubItem{内容命中:对于内容命中,除了样本挖掘和构造外,我们还提出了换域蒸馏等方案,解决了 60\% 的历史case}
  \item \textbf{模型结构优化}。针对不同场景下的匹配问题,我们对 ernie 模型的结构进行了优化
  \SubItem{PyramidErnie:针对匹配的特殊场景,我们结合双塔和单塔的特点,将 qt 先通过双塔 ernie 独立建模,再通过单塔 ernie 联合建模。结合 attention mask、词还原等方案,在参数量不变,计算量降低的情况下,PyramidErnie在所有数据集上指标均优于单塔 ernie}
  \SubItem{MOE: 整页不同阿拉丁卡片的匹配模式存在较大差异。针对此问题,我们在 ernie 基础上引入 moe,以卡片本身信息作为路由输入,pnr 相对提升 5\%}
\end{itemize}
\end{onehalfspacing}

\datedsubsection{\textbf{query 分析-改写}}{2018.7 -- 2020.5}
\role{微信搜一搜}{}
\begin{onehalfspacing}
在搜一搜主要负责 query 分析方向的改写,包括 term 改写 和 query 改写。
\begin{itemize}
  \item term 改写
  \SubItem{离线:\ 通过用户点击、session 切换等方式挖掘相似 query;在相似 query 基础上构造深度对齐模型产出对齐片段,训练 xgb 模型过滤低置信片段}
  \SubItem{在线:\ 利用标注数据,结合 PMI、词向量、QV、语言模型 等特征训练同义词后验 xgb 模型;在相关性模型上引入同义词特征
  }
  \item query 改写
  \SubItem{离线挖掘:利用 session、相似 query 等渠道挖掘改写数据,结合召回结果信息利用人工标注数据训练离线判别模型;同时结合用户点击率在线上进行二次清洗}
  \SubItem{模型生成:结合文本、拼音、字形等多种信息训练公众号改写数搜索生成式模型; 同时结合字、词、拼音、字形等特征学习向量化表示}
\end{itemize}
\end{onehalfspacing}

\end{document}